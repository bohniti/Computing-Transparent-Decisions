% !TeX spellcheck = en_GB
% !TeX encoding = UTF-8

\section*{Abstract}
\label{sec:abstract}

Nowadays, the decisions made by \gls{dl} Algorithms influence our everyday life. Since the Discussion around the "Directive on Copyright in the EU Digital Single Market" had started, more and more people are interested in the decisions which are made by algorithms. In the following, this thesis provides an theoretical introduction about \gls{ml}, \gls{dl} and \gls{xai}. Moreover, a prototype gives a realistic insight into how \gls{dl} and \gls{xai} can be implemented within a Jupyter Notebook. Finally, the achieved results get evaluated and generalized among the topics of \gls{ul}.

\section*{Preface I}
My work at the Siemens AG had given me tremendous insight into the business analytics field. But since my interested is going more into the statistics Field, I was highly motivated to work on topics related to  \gls{ml} and  \gls{dl}. In the next few years, I think of a position in the Data Science Field, and so I committed myself to a final year project which covers this field as well. The \gls{xai} area made me even more curious to find out what we can learn from machine learning algorithms. To work on this was a tough challenge,  but at least it was a useful insight into scientific writing. 

\section*{Preface II}

The first time I came into contact with Machine Learning was through an IT project. The intention was to develop a self-driving car, controlled by a deep neural network. The whole project was very challenging, but I had a lot of fun, and my motivation was even higher. My motivation was immediately put a rough test in my exchange semester. In Hong Kong, I focused on machine learning during the entire semester.  Mathematical proofs that were required to can pass one of the courses were an extremely challenging task. Mainly because of lacking prior knowledge of statistics, I had to work twice as hard. During the semester,  I was able to close the gaps, master the tasks and finally did an excellent job at the programming part. This experience has strengthened my wish to concentrate further in this direction, and so I was highly motived to put everything that I have learned into my final year project. Since then, I have learned a lot and even if the work was not as perfect as it could it helped me a lot particularly since I am participating in the computer science master program with a focus on data science. I am sure there is more to learn. So,  I think this thesis was a tiny step in my life long learning journey.